\documentclass[12pt]{article}

\usepackage[utf8]{inputenc}
\usepackage[russian]{babel}
\usepackage{amsmath,amssymb,amsthm,mathtools}
\usepackage{geometry}
\geometry{a4paper,margin=2.5cm}
\usepackage{hyperref}

\newtheorem{theorem}{Теорема}
\newtheorem{prop}{Утверждение}
\newtheorem{lemma}{Лемма}
\newtheorem{remark}{Замечание}

\DeclareMathOperator{\rank}{rank}

\title{Практическое занятие \#7.3. Метод Холецкого: теоретическое обоснование}
\author{}
\date{}

\begin{document}
\maketitle

\section*{1. Постановка}
Требуется решить систему линейных алгебраических уравнений
\[
A x = b,
\]
где матрица \(A\in\mathbb{R}^{n\times n}\) симметричная и, возможно, положительно определённая. Метод Холецкого основан на разложении
\[
A = L L^{\mathsf T},
\]
где \(L\) --- нижняя треугольная матрица с положительными диагональными элементами.

\section*{2. Условия существования разложения Холецкого}

\begin{theorem}[Классический критерий]
Для вещественной симметричной матрицы \(A\) следующие условия эквивалентны:
\begin{enumerate}
  \item \(A\) строго положительно определена: \(x^{\mathsf T} A x > 0\) для всех \(x\neq 0\).
  \item Существует единственное разложение Холецкого \(A = L L^{\mathsf T}\), где \(L\) нижнетреугольная и \(l_{ii}>0\).
  \item (Критерий Сильвестра) Все главные угловые миноры положительны.
\end{enumerate}
\end{theorem}

\begin{proof}[Идея доказательства]
\(1\Rightarrow 2\): индукция по размерности, на \(k\)-м шаге определяем
\[
l_{kk}=\sqrt{a_{kk}-\sum_{s=1}^{k-1} l_{ks}^2}>0,\qquad
l_{ik}=\frac{a_{ik}-\sum_{s=1}^{k-1} l_{is}l_{ks}}{l_{kk}},\ i>k,
\]
что возможно именно при положительной определённости (подкоренное \(>0\)).
Единственность следует из положительности диагонали.
\(2\Rightarrow 1\): \(x^{\mathsf T}A x = x^{\mathsf T} L L^{\mathsf T} x = \|L^{\mathsf T}x\|_2^2>0\) при \(x\neq 0\).
Эквивалентность с (3) --- классический результат Сильвестра. \end{proof}

\section*{3. Вычислительные формулы}
Из тождества \(A=L L^{\mathsf T}\) выводятся рекуррентные формулы:
\begin{align}
l_{kk} &= \sqrt{\,a_{kk}-\sum_{s=1}^{k-1} l_{ks}^2\,}, \label{eq:diag}\\
l_{ik} &= \frac{a_{ik}-\sum_{s=1}^{k-1} l_{is}l_{ks}}{l_{kk}},\quad i=k+1,\dots,n. \label{eq:off}
\end{align}
После построения \(L\) система решается последовательно:
\[
Ly=b \quad\text{(прямой ход)},\qquad L^{\mathsf T}x=y \quad\text{(обратный ход)}.
\]

\section*{4. Почему наш вариант \texttt{spd} корректен}
\begin{prop}\label{prop:spd}
Пусть \(M\in\mathbb{R}^{n\times n}\), \(\tau>0\) и \(A=M^{\mathsf T}M+\tau I\).
Тогда \(A\) симметрична и строго положительно определена.
\end{prop}
\begin{proof}
Симметрия очевидна. Для любого \(x\neq 0\):
\[
x^{\mathsf T}A x = x^{\mathsf T}M^{\mathsf T}M x + \tau x^{\mathsf T}x = \|Mx\|_2^2 + \tau \|x\|_2^2 > 0,
\]
так как \(\tau>0\) и \(\|x\|_2>0\). По теореме, для такого \(A\) существует \(A=LL^{\mathsf T}\), метод Холецкого применим.
\end{proof}

\section*{5. Почему вариант \texttt{bad} неприменим}
\begin{prop}\label{prop:bad}
Пусть \(A\) симметрична и имеет хотя бы одно неположительное собственное значение
(или, эквивалентно, хотя бы один неположительный главный угловой минор). Тогда \(A\) не строго положительно определена, и разложение Холецкого не существует.
\end{prop}
\begin{proof}
Если \(\lambda_{\min}\le 0\), то найдётся ненулевой вектор \(x\) с \(x^{\mathsf T}A x\le 0\). По теореме, положительная определённость нарушена, и формула \eqref{eq:diag} на некотором шаге даст
\(a_{kk}-\sum_{s=1}^{k-1} l_{ks}^2\le 0\), что делает \(\sqrt{\cdot}\) невозможным (в \(\mathbb{R}\)).
\end{proof}
В нашей программе \texttt{bad} строится как случайная симметричная матрица с искусственным понижением диагонали,
что с высокой вероятностью обеспечивает наличие неположительных главных миноров/собственных значений, и метод корректно «отказывает» на диагональном шаге.

\section*{6. Оценка числа операций}
\begin{lemma}[Трудоёмкость]
Построение \(A=LL^{\mathsf T}\) по \eqref{eq:diag}--\eqref{eq:off} требует порядка \(\frac{n^3}{3}\) арифметических операций
(без учёта \(n\) извлечений корня). Решение \(Ly=b\) и \(L^{\mathsf T}x=y\) занимает порядка \(2n^2\) операций.
Итого: \(\displaystyle \frac{n^3}{3}+2n^2\).
\end{lemma}
\begin{proof}[Идея подсчёта]
На \(k\)-м столбце суммирование по \(s=1,\dots,k-1\) даёт \(\sim k\) операций; суммарно по всем \(k\) получаем
\(\sum_{k=1}^{n} k \sim \tfrac{n(n+1)}{2}\) для каждого из вложенных циклов, что приводит к кубическому члену \(\frac{n^3}{3}\).
Треугольные ходы дают \(\sum_{i=1}^{n} i \sim \tfrac{n^2}{2}\) каждый, то есть порядка \(n^2\) каждый.
\end{proof}

\section*{7. Сравнение с методом Гаусса (для контекста)}
Для классического исключения Гаусса (без выбора главного элемента) ведущая трудоёмкость разложения составляет \(\frac{2}{3}n^3\) операций, плюс те же два треугольных хода \(\sim 2n^2\). Следовательно, для SPD-матриц метод Холецкого вдвое дешевле по кубическому члену.

\section*{8. Вывод}
\begin{itemize}
  \item Разложение Холецкого существует и единственно тогда и только тогда, когда \(A\) симметрична и строго положительно определена.
  \item Конструкция \(A=M^{\mathsf T}M+\tau I\) (\(\tau>0\)) гарантирует применимость метода.
  \item Симметричная, но не положительно определённая матрица неизбежно приводит к остановке алгоритма на диагональном шаге.
  \item Асимптотическая трудоёмкость: \(\frac{n^3}{3}+2n^2\) против \(\frac{2}{3}n^3+2n^2\) у метода Гаусса.
\end{itemize}



\end{document}